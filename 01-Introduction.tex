% Options for packages loaded elsewhere
\PassOptionsToPackage{unicode}{hyperref}
\PassOptionsToPackage{hyphens}{url}
%
\documentclass[
]{article}
\usepackage{lmodern}
\usepackage{amsmath}
\usepackage{ifxetex,ifluatex}
\ifnum 0\ifxetex 1\fi\ifluatex 1\fi=0 % if pdftex
  \usepackage[T1]{fontenc}
  \usepackage[utf8]{inputenc}
  \usepackage{textcomp} % provide euro and other symbols
  \usepackage{amssymb}
\else % if luatex or xetex
  \usepackage{unicode-math}
  \defaultfontfeatures{Scale=MatchLowercase}
  \defaultfontfeatures[\rmfamily]{Ligatures=TeX,Scale=1}
\fi
% Use upquote if available, for straight quotes in verbatim environments
\IfFileExists{upquote.sty}{\usepackage{upquote}}{}
\IfFileExists{microtype.sty}{% use microtype if available
  \usepackage[]{microtype}
  \UseMicrotypeSet[protrusion]{basicmath} % disable protrusion for tt fonts
}{}
\makeatletter
\@ifundefined{KOMAClassName}{% if non-KOMA class
  \IfFileExists{parskip.sty}{%
    \usepackage{parskip}
  }{% else
    \setlength{\parindent}{0pt}
    \setlength{\parskip}{6pt plus 2pt minus 1pt}}
}{% if KOMA class
  \KOMAoptions{parskip=half}}
\makeatother
\usepackage{xcolor}
\IfFileExists{xurl.sty}{\usepackage{xurl}}{} % add URL line breaks if available
\IfFileExists{bookmark.sty}{\usepackage{bookmark}}{\usepackage{hyperref}}
\hypersetup{
  pdftitle={Introduction et Concepts de base},
  hidelinks,
  pdfcreator={LaTeX via pandoc}}
\urlstyle{same} % disable monospaced font for URLs
\usepackage[margin=1in]{geometry}
\usepackage{color}
\usepackage{fancyvrb}
\newcommand{\VerbBar}{|}
\newcommand{\VERB}{\Verb[commandchars=\\\{\}]}
\DefineVerbatimEnvironment{Highlighting}{Verbatim}{commandchars=\\\{\}}
% Add ',fontsize=\small' for more characters per line
\usepackage{framed}
\definecolor{shadecolor}{RGB}{248,248,248}
\newenvironment{Shaded}{\begin{snugshade}}{\end{snugshade}}
\newcommand{\AlertTok}[1]{\textcolor[rgb]{0.94,0.16,0.16}{#1}}
\newcommand{\AnnotationTok}[1]{\textcolor[rgb]{0.56,0.35,0.01}{\textbf{\textit{#1}}}}
\newcommand{\AttributeTok}[1]{\textcolor[rgb]{0.77,0.63,0.00}{#1}}
\newcommand{\BaseNTok}[1]{\textcolor[rgb]{0.00,0.00,0.81}{#1}}
\newcommand{\BuiltInTok}[1]{#1}
\newcommand{\CharTok}[1]{\textcolor[rgb]{0.31,0.60,0.02}{#1}}
\newcommand{\CommentTok}[1]{\textcolor[rgb]{0.56,0.35,0.01}{\textit{#1}}}
\newcommand{\CommentVarTok}[1]{\textcolor[rgb]{0.56,0.35,0.01}{\textbf{\textit{#1}}}}
\newcommand{\ConstantTok}[1]{\textcolor[rgb]{0.00,0.00,0.00}{#1}}
\newcommand{\ControlFlowTok}[1]{\textcolor[rgb]{0.13,0.29,0.53}{\textbf{#1}}}
\newcommand{\DataTypeTok}[1]{\textcolor[rgb]{0.13,0.29,0.53}{#1}}
\newcommand{\DecValTok}[1]{\textcolor[rgb]{0.00,0.00,0.81}{#1}}
\newcommand{\DocumentationTok}[1]{\textcolor[rgb]{0.56,0.35,0.01}{\textbf{\textit{#1}}}}
\newcommand{\ErrorTok}[1]{\textcolor[rgb]{0.64,0.00,0.00}{\textbf{#1}}}
\newcommand{\ExtensionTok}[1]{#1}
\newcommand{\FloatTok}[1]{\textcolor[rgb]{0.00,0.00,0.81}{#1}}
\newcommand{\FunctionTok}[1]{\textcolor[rgb]{0.00,0.00,0.00}{#1}}
\newcommand{\ImportTok}[1]{#1}
\newcommand{\InformationTok}[1]{\textcolor[rgb]{0.56,0.35,0.01}{\textbf{\textit{#1}}}}
\newcommand{\KeywordTok}[1]{\textcolor[rgb]{0.13,0.29,0.53}{\textbf{#1}}}
\newcommand{\NormalTok}[1]{#1}
\newcommand{\OperatorTok}[1]{\textcolor[rgb]{0.81,0.36,0.00}{\textbf{#1}}}
\newcommand{\OtherTok}[1]{\textcolor[rgb]{0.56,0.35,0.01}{#1}}
\newcommand{\PreprocessorTok}[1]{\textcolor[rgb]{0.56,0.35,0.01}{\textit{#1}}}
\newcommand{\RegionMarkerTok}[1]{#1}
\newcommand{\SpecialCharTok}[1]{\textcolor[rgb]{0.00,0.00,0.00}{#1}}
\newcommand{\SpecialStringTok}[1]{\textcolor[rgb]{0.31,0.60,0.02}{#1}}
\newcommand{\StringTok}[1]{\textcolor[rgb]{0.31,0.60,0.02}{#1}}
\newcommand{\VariableTok}[1]{\textcolor[rgb]{0.00,0.00,0.00}{#1}}
\newcommand{\VerbatimStringTok}[1]{\textcolor[rgb]{0.31,0.60,0.02}{#1}}
\newcommand{\WarningTok}[1]{\textcolor[rgb]{0.56,0.35,0.01}{\textbf{\textit{#1}}}}
\usepackage{graphicx}
\makeatletter
\def\maxwidth{\ifdim\Gin@nat@width>\linewidth\linewidth\else\Gin@nat@width\fi}
\def\maxheight{\ifdim\Gin@nat@height>\textheight\textheight\else\Gin@nat@height\fi}
\makeatother
% Scale images if necessary, so that they will not overflow the page
% margins by default, and it is still possible to overwrite the defaults
% using explicit options in \includegraphics[width, height, ...]{}
\setkeys{Gin}{width=\maxwidth,height=\maxheight,keepaspectratio}
% Set default figure placement to htbp
\makeatletter
\def\fps@figure{htbp}
\makeatother
\setlength{\emergencystretch}{3em} % prevent overfull lines
\providecommand{\tightlist}{%
  \setlength{\itemsep}{0pt}\setlength{\parskip}{0pt}}
\setcounter{secnumdepth}{-\maxdimen} % remove section numbering
\ifluatex
  \usepackage{selnolig}  % disable illegal ligatures
\fi

\title{Introduction et Concepts de base}
\author{}
\date{\vspace{-2.5em}}

\begin{document}
\maketitle

\hypertarget{prise-en-main}{%
\section{Prise en Main}\label{prise-en-main}}

\hypertarget{comme-une-calculatrice}{%
\subsection{Comme une calculatrice:}\label{comme-une-calculatrice}}

\begin{itemize}
\tightlist
\item
  Vous voyez l'opérateur d'assignation \texttt{\textless{}-} :
\end{itemize}

\begin{Shaded}
\begin{Highlighting}[]
\NormalTok{a }\OtherTok{\textless{}{-}} \DecValTok{2}\SpecialCharTok{+}\DecValTok{2}
\NormalTok{b }\OtherTok{\textless{}{-}} \DecValTok{5{-}7}
\NormalTok{c }\OtherTok{\textless{}{-}} \DecValTok{4}\SpecialCharTok{*}\DecValTok{12}
\NormalTok{d }\OtherTok{\textless{}{-}} \DecValTok{10}\SpecialCharTok{/}\DecValTok{3}
\NormalTok{e }\OtherTok{\textless{}{-}} \DecValTok{5}\SpecialCharTok{\^{}}\DecValTok{2}
\end{Highlighting}
\end{Shaded}

\begin{Shaded}
\begin{Highlighting}[]
\NormalTok{a}
\end{Highlighting}
\end{Shaded}

\begin{verbatim}
## [1] 4
\end{verbatim}

\begin{Shaded}
\begin{Highlighting}[]
\NormalTok{b}
\end{Highlighting}
\end{Shaded}

\begin{verbatim}
## [1] -2
\end{verbatim}

\begin{Shaded}
\begin{Highlighting}[]
\NormalTok{c}
\end{Highlighting}
\end{Shaded}

\begin{verbatim}
## [1] 48
\end{verbatim}

\begin{Shaded}
\begin{Highlighting}[]
\NormalTok{d}
\end{Highlighting}
\end{Shaded}

\begin{verbatim}
## [1] 3.333333
\end{verbatim}

\begin{Shaded}
\begin{Highlighting}[]
\NormalTok{e}
\end{Highlighting}
\end{Shaded}

\begin{verbatim}
## [1] 25
\end{verbatim}

\begin{infobox}
\begin{itemize}
\item
  Les noms d'objets peuvent contenir des lettres, des chiffres, les
  symboles \texttt{.} et \texttt{\_}.
\item
  Ils ne peuvent pas commencer par un chiffre. Attention, R fait la
  différence entre minuscules et majuscules dans les noms d'objets, ce
  qui signifie que \texttt{x} et \texttt{X} seront deux objets
  différents, tout comme \texttt{resultat} et \texttt{Resultat}.
\end{itemize}

Conseils: - Il est préférable d'éviter les majuscules (pour les risques
d'erreur) et les caractères accentués (pour des questions d'encodage)
dans les noms d'objets.

\begin{itemize}
\tightlist
\item
  De même, il faut essayer de trouver un équilibre entre clarté du nom
  (comprendre à quoi sert l'objet, ce qu'il contient) et sa longueur.
  Par exemple, on préfèrera comme nom d'objet \texttt{taille\_conj1} à
  \texttt{taille\_du\_conjoint\_numero\_1} (trop long) ou à \texttt{t1}
  (pas assez explicite).
\end{itemize}
\end{infobox}

\hypertarget{cruxe9er-de-nouvelles-variables}{%
\subsubsection{Créer de nouvelles
variables}\label{cruxe9er-de-nouvelles-variables}}

\begin{Shaded}
\begin{Highlighting}[]
\NormalTok{Resultat }\OtherTok{\textless{}{-}}\NormalTok{ a }\SpecialCharTok{+}\NormalTok{ b}
\end{Highlighting}
\end{Shaded}

\hypertarget{objets-simples}{%
\subsection{Objets simples}\label{objets-simples}}

Types d'objets:

\begin{itemize}
\tightlist
\item
  Texte
\item
  Facteurs
\end{itemize}

\hypertarget{vecteurs}{%
\subsubsection{Vecteurs}\label{vecteurs}}

Imaginons maintenant qu'on a demandé la taille en centimètres de 5
personnes et qu'on souhaite calculer leur taille moyenne.

On pourrait créer autant d'objets que de tailles et faire l'opération
mathématique qui va bien :

\begin{Shaded}
\begin{Highlighting}[]
\NormalTok{taille1 }\OtherTok{\textless{}{-}} \DecValTok{156}
\NormalTok{taille2 }\OtherTok{\textless{}{-}} \DecValTok{164}
\NormalTok{taille3 }\OtherTok{\textless{}{-}} \DecValTok{197}
\NormalTok{taille4 }\OtherTok{\textless{}{-}} \DecValTok{147}
\NormalTok{taille5 }\OtherTok{\textless{}{-}} \DecValTok{173}
\NormalTok{(taille1 }\SpecialCharTok{+}\NormalTok{ taille2 }\SpecialCharTok{+}\NormalTok{ taille3 }\SpecialCharTok{+}\NormalTok{ taille4 }\SpecialCharTok{+}\NormalTok{ taille5) }\SpecialCharTok{/} \DecValTok{5}
\end{Highlighting}
\end{Shaded}

\begin{verbatim}
## [1] 167.4
\end{verbatim}

Cette manière de faire n'est évidemment pas pratique du tout. On va
plutôt stocker l'ensemble de nos tailles dans un seul objet, de type
\emph{vecteur}, avec la syntaxe suivante :

\begin{Shaded}
\begin{Highlighting}[]
\NormalTok{tailles }\OtherTok{\textless{}{-}} \FunctionTok{c}\NormalTok{(}\DecValTok{156}\NormalTok{, }\DecValTok{164}\NormalTok{, }\DecValTok{197}\NormalTok{, }\DecValTok{147}\NormalTok{, }\DecValTok{173}\NormalTok{)}
\end{Highlighting}
\end{Shaded}

Si on affiche le contenu de cet objet, on voit qu'il contient bien
l'ensemble des tailles saisies :

\begin{Shaded}
\begin{Highlighting}[]
\NormalTok{tailles}
\end{Highlighting}
\end{Shaded}

\begin{verbatim}
## [1] 156 164 197 147 173
\end{verbatim}

Un \emph{vecteur} dans R est un objet qui peut contenir plusieurs
informations du même type, potentiellement en très grand nombre.

L'avantage d'un vecteur est que lorsqu'on lui applique une opération,
celle-ci s'applique à toutes les valeurs qu'il contient. Ainsi, si on
veut la taille en mètres plutôt qu'en centimètres, on peut faire :

\begin{Shaded}
\begin{Highlighting}[]
\NormalTok{tailles\_m }\OtherTok{\textless{}{-}}\NormalTok{ tailles }\SpecialCharTok{/} \DecValTok{100}
\NormalTok{tailles\_m}
\end{Highlighting}
\end{Shaded}

\begin{verbatim}
## [1] 1.56 1.64 1.97 1.47 1.73
\end{verbatim}

\hypertarget{data-frames}{%
\subsection{Data Frames}\label{data-frames}}

\begin{Shaded}
\begin{Highlighting}[]
\NormalTok{df}\OtherTok{\textless{}{-}}\FunctionTok{data.frame}\NormalTok{(}\AttributeTok{subject=}\FunctionTok{as.factor}\NormalTok{(}\FunctionTok{c}\NormalTok{(}\StringTok{"Pierre"}\NormalTok{,}\StringTok{"Anne"}\NormalTok{,}\StringTok{"Joyce"}\NormalTok{,}\StringTok{"Peter"}\NormalTok{,}\StringTok{"Alan"}\NormalTok{,}\StringTok{"Camille"}\NormalTok{)), }\AttributeTok{age=}\FunctionTok{as.numeric}\NormalTok{(}\FunctionTok{c}\NormalTok{(}\DecValTok{20}\NormalTok{, }\DecValTok{16}\NormalTok{, }\DecValTok{19}\NormalTok{, }\DecValTok{99}\NormalTok{, }\DecValTok{23}\NormalTok{, }\DecValTok{18}\NormalTok{)), }\AttributeTok{sex=}\FunctionTok{as.factor}\NormalTok{(}\FunctionTok{c}\NormalTok{(}\StringTok{"M"}\NormalTok{,}\StringTok{"F"}\NormalTok{,}\StringTok{"F"}\NormalTok{,}\StringTok{"M"}\NormalTok{,}\ConstantTok{NA}\NormalTok{,}\StringTok{"F"}\NormalTok{)), }\AttributeTok{height=}\FunctionTok{as.numeric}\NormalTok{(}\FunctionTok{c}\NormalTok{(}\DecValTok{172}\NormalTok{, }\DecValTok{181}\NormalTok{, }\DecValTok{165}\NormalTok{, }\DecValTok{168}\NormalTok{, }\DecValTok{177}\NormalTok{, }\DecValTok{178}\NormalTok{)), }\AttributeTok{speed=}\FunctionTok{as.numeric}\NormalTok{(}\FunctionTok{c}\NormalTok{(}\FloatTok{11.20}\NormalTok{,}\FloatTok{3.00}\NormalTok{,}\FloatTok{11.50}\NormalTok{,}\FloatTok{10.35}\NormalTok{,}\FloatTok{10.98}\NormalTok{,}\FloatTok{13.05}\NormalTok{)))}
\end{Highlighting}
\end{Shaded}

Check the data.frame df by simply typing in:

\begin{Shaded}
\begin{Highlighting}[]
\NormalTok{df}
\end{Highlighting}
\end{Shaded}

\begin{verbatim}
##   subject age  sex height speed
## 1  Pierre  20    M    172 11.20
## 2    Anne  16    F    181  3.00
## 3   Joyce  19    F    165 11.50
## 4   Peter  99    M    168 10.35
## 5    Alan  23 <NA>    177 10.98
## 6 Camille  18    F    178 13.05
\end{verbatim}

The data has 6 rows and 5 columns:

\begin{Shaded}
\begin{Highlighting}[]
\FunctionTok{dim}\NormalTok{(df)}
\end{Highlighting}
\end{Shaded}

\begin{verbatim}
## [1] 6 5
\end{verbatim}

Because the data are not that big you are able to view them entirely in
your R studio. However, if you have very big data you may want to view
only the first lines:

\begin{Shaded}
\begin{Highlighting}[]
\FunctionTok{head}\NormalTok{(df, }\DecValTok{4}\NormalTok{)}
\end{Highlighting}
\end{Shaded}

\begin{verbatim}
##   subject age sex height speed
## 1  Pierre  20   M    172 11.20
## 2    Anne  16   F    181  3.00
## 3   Joyce  19   F    165 11.50
## 4   Peter  99   M    168 10.35
\end{verbatim}

To inspect the structure of the data:

\begin{Shaded}
\begin{Highlighting}[]
\FunctionTok{str}\NormalTok{(df)}
\end{Highlighting}
\end{Shaded}

\begin{verbatim}
## 'data.frame':    6 obs. of  5 variables:
##  $ subject: Factor w/ 6 levels "Alan","Anne",..: 6 2 4 5 1 3
##  $ age    : num  20 16 19 99 23 18
##  $ sex    : Factor w/ 2 levels "F","M": 2 1 1 2 NA 1
##  $ height : num  172 181 165 168 177 178
##  $ speed  : num  11.2 3 11.5 10.3 11 ...
\end{verbatim}

Some basic statistics can be given by the summary command:

\begin{Shaded}
\begin{Highlighting}[]
\FunctionTok{summary}\NormalTok{(df)}
\end{Highlighting}
\end{Shaded}

\begin{verbatim}
##     subject       age          sex        height          speed      
##  Alan   :1   Min.   :16.00   F   :3   Min.   :165.0   Min.   : 3.00  
##  Anne   :1   1st Qu.:18.25   M   :2   1st Qu.:169.0   1st Qu.:10.51  
##  Camille:1   Median :19.50   NA's:1   Median :174.5   Median :11.09  
##  Joyce  :1   Mean   :32.50            Mean   :173.5   Mean   :10.01  
##  Peter  :1   3rd Qu.:22.25            3rd Qu.:177.8   3rd Qu.:11.43  
##  Pierre :1   Max.   :99.00            Max.   :181.0   Max.   :13.05
\end{verbatim}

\hypertarget{missing-values}{%
\subsubsection{Missing values}\label{missing-values}}

To examine if missing values (NA) are in the data:

\begin{Shaded}
\begin{Highlighting}[]
\FunctionTok{is.na}\NormalTok{(df)}
\end{Highlighting}
\end{Shaded}

\begin{verbatim}
##      subject   age   sex height speed
## [1,]   FALSE FALSE FALSE  FALSE FALSE
## [2,]   FALSE FALSE FALSE  FALSE FALSE
## [3,]   FALSE FALSE FALSE  FALSE FALSE
## [4,]   FALSE FALSE FALSE  FALSE FALSE
## [5,]   FALSE FALSE  TRUE  FALSE FALSE
## [6,]   FALSE FALSE FALSE  FALSE FALSE
\end{verbatim}

If you want the row and column index where NAs occur:

\begin{Shaded}
\begin{Highlighting}[]
\FunctionTok{which}\NormalTok{(}\FunctionTok{is.na}\NormalTok{(df), }\AttributeTok{arr.ind=}\NormalTok{T)}
\end{Highlighting}
\end{Shaded}

\begin{verbatim}
##      row col
## [1,]   5   3
\end{verbatim}

\hypertarget{impossible-extreme-values}{%
\subsubsection{Impossible extreme
values}\label{impossible-extreme-values}}

We can use boxplots to see if there are impossible extreme values:

\begin{Shaded}
\begin{Highlighting}[]
\FunctionTok{par}\NormalTok{(}\AttributeTok{mfrow=}\FunctionTok{c}\NormalTok{(}\DecValTok{1}\NormalTok{,}\DecValTok{3}\NormalTok{))}
\FunctionTok{boxplot}\NormalTok{(df}\SpecialCharTok{$}\NormalTok{age, }\AttributeTok{main=}\StringTok{"Age (yrs)"}\NormalTok{, }\AttributeTok{cex.lab=}\FloatTok{2.0}\NormalTok{, }\AttributeTok{cex.axis=}\FloatTok{2.0}\NormalTok{, }\AttributeTok{cex.main=}\FloatTok{1.6}\NormalTok{, }\AttributeTok{cex=}\FloatTok{2.0}\NormalTok{, }\AttributeTok{col=}\StringTok{"yellow"}\NormalTok{)}
\FunctionTok{boxplot}\NormalTok{(df}\SpecialCharTok{$}\NormalTok{height, }\AttributeTok{main=}\StringTok{"Height (cm)"}\NormalTok{, }\AttributeTok{cex.lab=}\FloatTok{2.0}\NormalTok{, }\AttributeTok{cex.axis=}\FloatTok{2.0}\NormalTok{, }\AttributeTok{cex.main=}\FloatTok{1.6}\NormalTok{, }\AttributeTok{cex=}\FloatTok{2.0}\NormalTok{, }\AttributeTok{col=}\StringTok{"red"}\NormalTok{)}
\FunctionTok{boxplot}\NormalTok{(df}\SpecialCharTok{$}\NormalTok{speed, }\AttributeTok{main=}\StringTok{"Speed (ms)"}\NormalTok{, }\AttributeTok{cex.lab=}\FloatTok{2.0}\NormalTok{, }\AttributeTok{cex.axis=}\FloatTok{2.0}\NormalTok{, }\AttributeTok{cex.main=}\FloatTok{1.6}\NormalTok{, }\AttributeTok{cex=}\FloatTok{2.0}\NormalTok{, }\AttributeTok{col=}\StringTok{"orange"}\NormalTok{)}
\end{Highlighting}
\end{Shaded}

\includegraphics{01-Introduction_files/figure-latex/unnamed-chunk-18-1.pdf}

\hypertarget{finding-replacing-the-extreme-values-manually}{%
\subsubsection{Finding \& replacing the extreme values
manually}\label{finding-replacing-the-extreme-values-manually}}

Compute the mean \emph{before} the removal of outliers:

\begin{Shaded}
\begin{Highlighting}[]
\FunctionTok{mean}\NormalTok{(df}\SpecialCharTok{$}\NormalTok{age)}
\end{Highlighting}
\end{Shaded}

\begin{verbatim}
## [1] 32.5
\end{verbatim}

Check if there are cases that are older than 40 years:

\begin{Shaded}
\begin{Highlighting}[]
\NormalTok{df}\SpecialCharTok{$}\NormalTok{age}\SpecialCharTok{\textgreater{}}\DecValTok{40}
\end{Highlighting}
\end{Shaded}

\begin{verbatim}
## [1] FALSE FALSE FALSE  TRUE FALSE FALSE
\end{verbatim}

Replace the case(s) older than 40 with a missing value (NA).

\begin{Shaded}
\begin{Highlighting}[]
\NormalTok{df}\SpecialCharTok{$}\NormalTok{age[df}\SpecialCharTok{$}\NormalTok{age}\SpecialCharTok{\textgreater{}}\DecValTok{40}\NormalTok{]}\OtherTok{\textless{}{-}}\ConstantTok{NA}
\end{Highlighting}
\end{Shaded}

Compute again the mean age, allowing to remove missing values (NAs):

\begin{Shaded}
\begin{Highlighting}[]
\FunctionTok{mean}\NormalTok{(df}\SpecialCharTok{$}\NormalTok{age, }\AttributeTok{na.rm=}\ConstantTok{TRUE}\NormalTok{)}
\end{Highlighting}
\end{Shaded}

\begin{verbatim}
## [1] 19.2
\end{verbatim}

\hypertarget{inspecting-mean-age-for-male-and-female-participants}{%
\subsubsection{Inspecting mean age for male and female
participants}\label{inspecting-mean-age-for-male-and-female-participants}}

Mean age for male and female participants.

\begin{Shaded}
\begin{Highlighting}[]
\FunctionTok{aggregate}\NormalTok{(age }\SpecialCharTok{\textasciitilde{}}\NormalTok{ sex, }\AttributeTok{data=}\NormalTok{df, }\AttributeTok{FUN=}\NormalTok{mean, }\AttributeTok{na.rm=}\ConstantTok{TRUE}\NormalTok{)}
\end{Highlighting}
\end{Shaded}

\begin{verbatim}
##   sex      age
## 1   F 17.66667
## 2   M 20.00000
\end{verbatim}

\hypertarget{inspecting-relations-between-variables}{%
\subsubsection{Inspecting relations between
variables}\label{inspecting-relations-between-variables}}

Use a scatterplot to display the relation between age and speed:

\begin{Shaded}
\begin{Highlighting}[]
\FunctionTok{par}\NormalTok{(}\AttributeTok{mfrow=}\FunctionTok{c}\NormalTok{(}\DecValTok{1}\NormalTok{,}\DecValTok{1}\NormalTok{))}
\FunctionTok{plot}\NormalTok{(speed }\SpecialCharTok{\textasciitilde{}}\NormalTok{ age, }\AttributeTok{data=}\NormalTok{df, }\AttributeTok{col=}\StringTok{"blue"}\NormalTok{, }\AttributeTok{pch=}\DecValTok{1}\NormalTok{, }\AttributeTok{cex=}\FloatTok{1.2}\NormalTok{)}
\end{Highlighting}
\end{Shaded}

\includegraphics{01-Introduction_files/figure-latex/unnamed-chunk-25-1.pdf}

\hypertarget{quiz-questions-and-answers}{%
\section{\texorpdfstring{{QUIZ QUESTIONS AND
ANSWERS}}{QUIZ QUESTIONS AND ANSWERS}}\label{quiz-questions-and-answers}}

\hypertarget{quiz-question-i}{%
\subsubsection{Quiz Question I}\label{quiz-question-i}}

Replace the outlier of 3.00 ms in the variable \emph{df\$speed} with a
NA.

\textbf{Answer} There are multiple solutions possible:

\begin{Shaded}
\begin{Highlighting}[]
\NormalTok{df}\SpecialCharTok{$}\NormalTok{speed[df}\SpecialCharTok{$}\NormalTok{speed}\SpecialCharTok{==}\FloatTok{3.00}\NormalTok{]}\OtherTok{\textless{}{-}}\ConstantTok{NA}
\end{Highlighting}
\end{Shaded}

Another solution:

\begin{Shaded}
\begin{Highlighting}[]
\NormalTok{df[}\DecValTok{2}\NormalTok{,}\DecValTok{5}\NormalTok{]}\OtherTok{\textless{}{-}}\ConstantTok{NA}
\end{Highlighting}
\end{Shaded}

To verify that the value was indeed replaced by a NA use the command
\texttt{is.na}:

\begin{Shaded}
\begin{Highlighting}[]
\FunctionTok{is.na}\NormalTok{(df}\SpecialCharTok{$}\NormalTok{speed)}
\end{Highlighting}
\end{Shaded}

\begin{verbatim}
## [1] FALSE  TRUE FALSE FALSE FALSE FALSE
\end{verbatim}

\hypertarget{quiz-question-ii}{%
\subsubsection{Quiz Question II}\label{quiz-question-ii}}

A dataset could erroneously have double records (duplicates). This is
bad and should be removed.

Seeing the current dataset, what would be a way to discover duplicates
in the variable \emph{df\$subject}?

Check your solution for the following data.frame that contains double
data:

\begin{Shaded}
\begin{Highlighting}[]
\NormalTok{df}\OtherTok{\textless{}{-}}\FunctionTok{data.frame}\NormalTok{(}\AttributeTok{subject=}\FunctionTok{as.factor}\NormalTok{(}\FunctionTok{c}\NormalTok{(}\StringTok{"Pierre"}\NormalTok{,}\StringTok{"Anne"}\NormalTok{,}\StringTok{"Joyce"}\NormalTok{,}\StringTok{"Peter"}\NormalTok{,}\StringTok{"Alan"}\NormalTok{,}\StringTok{"Camille"}\NormalTok{, }\StringTok{"Pierre"}\NormalTok{)), }\AttributeTok{age=}\FunctionTok{as.numeric}\NormalTok{(}\FunctionTok{c}\NormalTok{(}\DecValTok{20}\NormalTok{, }\DecValTok{16}\NormalTok{, }\DecValTok{19}\NormalTok{, }\DecValTok{99}\NormalTok{, }\DecValTok{23}\NormalTok{, }\DecValTok{18}\NormalTok{, }\DecValTok{20}\NormalTok{)), }\AttributeTok{sex=}\FunctionTok{as.factor}\NormalTok{(}\FunctionTok{c}\NormalTok{(}\StringTok{"M"}\NormalTok{,}\StringTok{"F"}\NormalTok{,}\StringTok{"F"}\NormalTok{,}\StringTok{"M"}\NormalTok{,}\ConstantTok{NA}\NormalTok{,}\StringTok{"F"}\NormalTok{, }\StringTok{"M"}\NormalTok{)), }\AttributeTok{height=}\FunctionTok{as.numeric}\NormalTok{(}\FunctionTok{c}\NormalTok{(}\DecValTok{172}\NormalTok{, }\DecValTok{181}\NormalTok{, }\DecValTok{165}\NormalTok{, }\DecValTok{168}\NormalTok{, }\DecValTok{177}\NormalTok{, }\DecValTok{178}\NormalTok{, }\DecValTok{172}\NormalTok{)), }\AttributeTok{speed=}\FunctionTok{as.numeric}\NormalTok{(}\FunctionTok{c}\NormalTok{(}\FloatTok{11.20}\NormalTok{,}\FloatTok{3.00}\NormalTok{,}\FloatTok{11.50}\NormalTok{,}\FloatTok{10.35}\NormalTok{,}\FloatTok{10.98}\NormalTok{,}\FloatTok{13.05}\NormalTok{, }\FloatTok{11.20}\NormalTok{)))}
\end{Highlighting}
\end{Shaded}

\textbf{Answer}

A way to inspect for double records is to use the \texttt{table}
function.

\begin{Shaded}
\begin{Highlighting}[]
\FunctionTok{table}\NormalTok{(df}\SpecialCharTok{$}\NormalTok{subject)}
\end{Highlighting}
\end{Shaded}

\begin{verbatim}
## 
##    Alan    Anne Camille   Joyce   Peter  Pierre 
##       1       1       1       1       1       2
\end{verbatim}

But still simpler, using the \texttt{summary} function would also
display this:

\begin{Shaded}
\begin{Highlighting}[]
\FunctionTok{summary}\NormalTok{(df)}
\end{Highlighting}
\end{Shaded}

\begin{verbatim}
##     subject       age          sex        height          speed      
##  Alan   :1   Min.   :16.00   F   :3   Min.   :165.0   Min.   : 3.00  
##  Anne   :1   1st Qu.:18.50   M   :3   1st Qu.:170.0   1st Qu.:10.66  
##  Camille:1   Median :20.00   NA's:1   Median :172.0   Median :11.20  
##  Joyce  :1   Mean   :30.71            Mean   :173.3   Mean   :10.18  
##  Peter  :1   3rd Qu.:21.50            3rd Qu.:177.5   3rd Qu.:11.35  
##  Pierre :2   Max.   :99.00            Max.   :181.0   Max.   :13.05
\end{verbatim}

\begin{Shaded}
\begin{Highlighting}[]
\FunctionTok{library}\NormalTok{(questionr)}
\FunctionTok{data}\NormalTok{(hdv2003)}
\end{Highlighting}
\end{Shaded}


\end{document}
